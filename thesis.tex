\documentclass[
11pt, % The default document font size, options: 10pt, 11pt, 12pt
%oneside, % Two side (alternating margins) for binding by default, uncomment to switch to one side
english, % ngerman for German
singlespacing, % Single line spacing, alternatives: onehalfspacing or doublespacing
%draft, % Uncomment to enable draft mode (no pictures, no links, overfull hboxes indicated)
%nolistspacing, % If the document is onehalfspacing or doublespacing, uncomment this to set spacing in lists to single
%liststotoc, % Uncomment to add the list of figures/tables/etc to the table of contents
%toctotoc, % Uncomment to add the main table of contents to the table of contents
%parskip, % Uncomment to add space between paragraphs
%nohyperref, % Uncomment to not load the hyperref package
headsepline, % Uncomment to get a line under the header
%chapterinoneline, % Uncomment to place the chapter title next to the number on one line
%consistentlayout, % Uncomment to change the layout of the declaration, abstract and acknowledgements pages to match the default layout
]{MastersDoctoralThesis} % The class file specifying the document structure

\usepackage[utf8]{inputenc} % Required for inputting international characters
\usepackage[T1]{fontenc} % Output font encoding for international characters

\usepackage{mathpazo} % Use the Palatino font by default

\usepackage[backend=bibtex,style=authoryear,natbib=true]{biblatex} % Use the bibtex backend with the authoryear citation style (which resembles APA)

\addbibresource{literature.bib} % The filename of the bibliography

\usepackage[autostyle=true]{csquotes} % Required to generate language-dependent quotes in the bibliography

%----------------------------------------------------------------------------------------
%	MARGIN SETTINGS
%----------------------------------------------------------------------------------------

\geometry{
	paper=a4paper, % Change to letterpaper for US letter
	inner=2.5cm, % Inner margin
	outer=3.8cm, % Outer margin
	bindingoffset=.5cm, % Binding offset
	top=1.5cm, % Top margin
	bottom=1.5cm, % Bottom margin
	%showframe, % Uncomment to show how the type block is set on the page
}

%----------------------------------------------------------------------------------------
%	THESIS INFORMATION
%----------------------------------------------------------------------------------------

\thesistitle{Speech-Processing-Pipeline to overcome language barriers in real-time communication} % Your thesis title, this is used in the title and abstract, print it elsewhere with \ttitle
\psupervisor{Prof. Dr. Martin Rumpler} % Your primary supervisor's name, this is used in the title page, print it elsewhere with \psupname
\ssupervisor{Prof. Dr. Klaus-Uwe Gollmer} % Your secondary supervisor's name, this is used in the title page, print it elsewhere with \ssupname
\examiner{} % Your examiner's name, this is not currently used anywhere in the template, print it elsewhere with \examname
\degree{Bachelor of Science (B. Sc.)} % Your degree name, this is used in the title page and abstract, print it elsewhere with \degreename
\author{Jason Rietzke} % Your name, this is used in the title page and abstract, print it elsewhere with \authorname
\addresses{} % Your address, this is not currently used anywhere in the template, print it elsewhere with \addressname

\subject{Computer Science} % Your subject area, this is not currently used anywhere in the template, print it elsewhere with \subjectname
\keywords{} % Keywords for your thesis, this is not currently used anywhere in the template, print it elsewhere with \keywordnames
\university{\href{https://umwelt-campus.de}{Trier University of Applied Sciences, Environmental Campus Birkenfeld}} % Your university's name and URL, this is used in the title page and abstract, print it elsewhere with \univname
\department{\href{https://www.umwelt-campus.de/en/study/study-programmes-continuing-education/bachelor/environmental-informatics-and-business-information-systems-bsc}{Environmental Informatics and Business Information Systems}} % Your department's name and URL, this is used in the title page and abstract, print it elsewhere with \deptname
\group{} % Your research group's name and URL, this is used in the title page, print it elsewhere with \groupname
\faculty{Umweltplanung / Umwelttechnik} % Your faculty's name and URL, this is used in the title page and abstract, print it elsewhere with \facname

\AtBeginDocument{
\hypersetup{pdftitle=\ttitle} % Set the PDF's title to your title
\hypersetup{pdfauthor=\authorname} % Set the PDF's author to your name
\hypersetup{pdfkeywords=\keywordnames} % Set the PDF's keywords to your keywords
}

\begin{document}

\frontmatter % Use roman page numbering style (i, ii, iii, iv...) for the pre-content pages

\pagestyle{plain} % Default to the plain heading style until the thesis style is called for the body content

%----------------------------------------------------------------------------------------
%	TITLE PAGE
%----------------------------------------------------------------------------------------

\begin{titlepage}
	\begin{center}
		{\huge \bfseries \ttitle\par}\vspace{0.4cm}
		\HRule \\[1.5cm]

		\begin{minipage}[t]{0.4\textwidth}
		\begin{flushleft} \large
		\emph{Author:}\\
		\authorname\\[0.3cm]
		\emph{Semester:}\\
		{9th Semester}\\[0.3cm]
		\emph{Date of Submission:}\\
		{January 9th, 2024}\\[0.3cm]

		\end{flushleft}
		\end{minipage}
		\begin{minipage}[t]{0.5\textwidth}
		\begin{flushright} \large
		\emph{Primary Supervisor:} \\
		\psupname\\[0.3cm]
		\emph{Secondary Supervisor:} \\
		\ssupname\\
		\end{flushright}
		\end{minipage}\\[3cm]
		
		\vfill

		\large\textit{
			A bachelor's thesis submitted in fulfillment of the requirements\\ for the degree of \degreename
		}\\[0.3cm]
		\large\textit{in the}\\[0.4cm]
		\large\textit\deptname\\[0.4cm]
		\large\textit{at the}\\[0.4cm]
		\large\textit\univname\\[2cm]

		\vfill
	\end{center}
\end{titlepage}

%----------------------------------------------------------------------------------------
%	DECLARATION PAGE
%----------------------------------------------------------------------------------------

\begin{declaration}
\addchaptertocentry{\authorshipname} % Add the declaration to the table of contents
\noindent I, \authorname, declare that this thesis titled, \enquote{\ttitle} and the work presented in it are written 
by myself and that I did not use any other aids and sources than those indicated. All text passages that have been 
taken verbatim or in spirit from other works are marked as such. The work has not been submitted to any other reviewing 
body in the same or a similar form.
\\

\noindent Date: Oberthal, Germany - January 9th, 2024\\
\rule[0.5em]{25em}{0.5pt}

\noindent Signed:\\
\rule[0.5em]{25em}{0.5pt}\\
\noindent {\authorname}\\

\end{declaration}

\cleardoublepage

%----------------------------------------------------------------------------------------
%	ABSTRACT PAGE
%----------------------------------------------------------------------------------------

\begin{abstract}
\addchaptertocentry{\abstractname}

The primary objective of this bachelor's thesis is to develop a testable prototype for a speech-processing pipeline 
that receives audio streams from various sources and transcribes and translates them in near real-time. The prototype 
facilitates communication between emergency control centers and callers speaking different languages. The focus is on 
utilizing open-source technologies like OpenAI Whisper for speech recognition. The System's functionality revolves 
around processing audio streams, segmenting them into individual messages, and managing them within a session context 
specific to foreign language communication. The prototype is integrated into the Notitia application, leveraging 
existing internal modules for user authentication and data-stream integration to the Voice over IP provider.

The thesis explores the integration of various open-source technologies. It aims to optimize the processing time of the 
speech-processing pipeline to meet the stringent time constraints that characterize emergencies. The average processing 
time of the pipeline is measured for different numbers of simultaneous audio streams to evaluate the performance and 
conclude a sufficient transcription interval.

This work shows that the speech-processing pipeline meets most requirements and is sufficiently fast to handle multiple 
audio streams simultaneously. It also shows that the pipeline can be built using primarily open-source software and 
commodity hardware. The text translation is the only component not based on an open-source project.

This thesis concludes that the speech-processing pipeline is a viable solution for emergency call centers to overcome 
language barriers in real-time communication. It can provide near real-time transcription and translation of spoken 
content and can be deployed on-premise to ensure reliability in emergencies.

\end{abstract}

%----------------------------------------------------------------------------------------
%	LIST OF CONTENTS/FIGURES/TABLES PAGES
%----------------------------------------------------------------------------------------

\tableofcontents % Prints the main table of contents

\listoffigures % Prints the list of figures

\listoftables % Prints the list of tables

%----------------------------------------------------------------------------------------
%	ABBREVIATIONS
%----------------------------------------------------------------------------------------

\begin{abbreviations}{ll} % Include a list of abbreviations (a table of two columns)

\end{abbreviations}

%----------------------------------------------------------------------------------------
%	THESIS CONTENT - CHAPTERS
%----------------------------------------------------------------------------------------

\mainmatter % Begin numeric (1,2,3...) page numbering

\pagestyle{thesis} % Return the page headers back to the "thesis" style

% Include the chapters of the thesis as separate files from the Chapters folder
% Uncomment the lines as you write the chapters

\chapter{Introduction}

\label{Introduction}

%----------------------------------------------------------------------------------------

% Define some commands to keep the formatting separated from the content 
\newcommand{\keyword}[1]{\textbf{#1}}
\newcommand{\tabhead}[1]{\textbf{#1}}
\newcommand{\code}[1]{\texttt{#1}}
\newcommand{\file}[1]{\texttt{\bfseries#1}}
\newcommand{\option}[1]{\texttt{\itshape#1}}

%----------------------------------------------------------------------------------------

\section{Background \& Motivation}

% This work is conducted in the context of the Notitia Suite of the LiveReader GmbH, within the SPELL research project. 
% The SPELL research projects aims to improve the software of critical infrastructures, such as emergency control 
% centers. In this context the Notitia Suite serves the prupose of a interface for the emergency calltaker to dispatch 
% the emergency services. \\
% \\
% An important part of dispatching the proper resources is to understand the needs of the emergency caller. 
% This is especially difficult when the caller and the calltaker do not speak the same language and important details 
% can not be conveyed properly. \\
% \\
% This work aims to develop a speech-processing pipeline that can be integrated into the Notitia Suite to allow for 
% communication between the caller and the calltaker speaking foreign languages. The pipeline itself is able to work 
% stand alone and can be integrated into other applications but does not have to be. \\
% This work will focus on the development of the speech-processing pipeline from here on and won't go into detail 
% regarding intigration into the Notitia Suite or any other application that may use it as a service.

Effective communication is critical in emergency services, where timely and accurate information is vital to 
saving lives. The Notitia application, within the context of the SPELL research project, aims to establish a 
communication framework between emergency control centers and callers. The project envisions a seamless integration of 
a speech-processing pipeline to transcribe and translate audio streams in real-time, addressing the language barriers 
that often impede clear communication during emergencies.

The SPELL research project recognizes the importance of near real-time transcription and translation services in empowering emergency control centers—existing technologies present challenges that need innovative solutions. 
Many systems designed for continuous audio streams are cloud-based, introducing concerns regarding reliability, 
latency, and request priority—critical factors in emergency services. The proposed solution aims to 
overcome these challenges by leveraging open-source technologies and building a testable prototype that can be deployed 
on-premise, ensuring reliability in time-sensitive scenarios.

%----------------------------------------------------------------------------------------

\section{Problem Statement}

% Despite the advancements in speech processing, existing solutions for transcription and translation in continuous audio 
% streams often fall short of meeting the specific demands of emergency call centers. Cloud-based systems, while widely 
% available, pose risks related to their reliability and latency, which are critical considerations in the context of 
% emergency response. The inability to seamlessly integrate features from different providers further hinders the 
% optimization of the transcription and translation process.\\
% \\
% The challenge, therefore, is to develop a speech-processing pipeline that not only addresses the language barriers in 
% emergency communication but also meets the stringent requirements of reliability, low latency, and on-premise 
% deployment. The integration of OpenAI Whisper, DeepL, and PiperTTS as foundational technologies sets the stage for a 
% comprehensive solution, yet the task is to weave these components into a coherent system tailored for emergency 
% services.

Despite the rapid advancements in speech processing technologies, the unique demands of emergency call centers pose 
significant challenges that current solutions need help to overcome. The SPELL research project identifies a crucial gap 
in systems designed for continuous audio streams. While cloud-based platforms offer convenience and 
scalability, they introduce a host of concerns when applied to the critical infrastructure of emergency services. 
Issues such as reliability, latency, and request prioritization become magnified in time-sensitive scenarios, 
jeopardizing the effectiveness of emergency responses.

The reliance on cloud solutions also restricts the flexibility to deploy services on-premise, a key consideration for 
maintaining the operational integrity of emergency control centers. Moreover, the homogeneous nature of many 
speech-processing systems impedes the seamless integration of features from diverse providers. This limitation prevents 
emergency services from harnessing the full spectrum of capabilities offered by various tools, hindering the 
optimization of transcription and translation processes.

In essence, the challenge is not just about overcoming language barriers; it is about tailoring a solution that aligns 
with the unique requirements of emergency communication. The proposed speech-processing pipeline must be a fusion of 
innovation and practicality, addressing linguistic diversity and the critical need for reliability, low 
latency, and the ability to operate on-premise.

Therefore, the problem is multifaceted: How can a speech processing pipeline be meticulously crafted to 
transcend the limitations of current technologies, providing a reliable, low-latency, and on-premise solution for 
emergency call centers? This challenge necessitates an exploration of open-source technologies, strategic integration 
of proven components like OpenAI Whisper and DeepL, and the development of a tailored prototype, all within the 
stringent time constraints that characterize emergencies. The task is to navigate the intricacies of emergency 
communication, ensuring that the proposed solution becomes an indispensable tool for effective and timely response in 
the face of linguistic diversity.


%----------------------------------------------------------------------------------------

\section{Objectives \& Scope}

The primary objective of this bachelor's thesis is to develop a testable prototype for a speech-processing pipeline 
within the Notitia application. This prototype aims to transcribe and translate audio streams in near real-time, 
facilitating communication between emergency control centers and callers speaking different languages. The focus is on 
utilizing existing open-source technologies, such as OpenAI Whisper for speech recognition, DeepL for translation, and 
PiperTTS for speech synthesis.

The scope encompasses the integration of the speech-processing pipeline into the Notitia application, leveraging 
existing internal modules for user authentication and data-stream integration to the Voice over IP provider. The 
system's functionality revolves around processing audio streams, segmenting them into individual messages, and managing 
them within a session context specific to foreign language communication.

%----------------------------------------------------------------------------------------

\section{Research Questions \& Methodology}

\subsection{Research Questions}

How can a speech processing pipeline, predominantly built upon open-source software, be harnessed to facilitate near 
real-time transcription and translation of spoken content to overcome language barriers in the context of an emergency 
call center?

\subsection{Methodology}

The research methodology involves the development of a testable prototype with a focus on minimizing processing time. 
The integration of OpenAI Whisper, DeepL, and PiperTTS will be explored to optimize the transcription and translation 
process. The system will utilize existing internal modules for user authentication and data-stream integration, with a 
data stream transferring audio in WAV format. The research emphasizes on-premise deployment for reliability in 
emergency scenarios.

In the subsequent chapters, we delve into the technical details of the speech-processing pipeline, its integration into 
Notitia, and the evaluation of its performance in the critical context of emergency communication.

%----------------------------------------------------------------------------------------

\section{Development Process}

For the development of this project, Visual Studio Code is the IDE of choice. The project utilizes a microservice 
architecture, which allows the development of different system parts independently and deploying them separately. 
The project is versioned using Git and hosted privately on GitHub under the LiveReader organization. 

The primary runtime for the backend is Node.js version 20 LTS. It provides an API interface for any frontend or third 
party to communicate with.
The web interface uses Vue.js as a UI framework, extended by Vuetify as a UI component library.

\chapter{Concept}

\label{Chapter2}

%----------------------------------------------------------------------------------------

\section{Audio Input}

The flexibility of the system allows for various audio input options. This work will cover two of the most important 
ones. The two most critical audio sources are recordings by web browsers and audio input provided by voice-over-IP 
communication systems.

\subsection{Web Browser Audio}

Within the context of this work, the focus lies on integrating Chromium-based web browsers. The Web MediaRecorder API 
provides the audio input. The API requests access to the microphone and allows for recording audio streams. 
The audio stream gets sent to the server as WebM-encoded data chunks of about 200ms. The server then concatenates the 
chunks and converts the audio stream to WAV format. The conversion utilizes FFmpeg, a command-line tool for 
manipulating audio and video files. The speech-processing pipeline then processes the resulting WAV file.

\subsection{Voice-over-IP Audio}

To get access to VoIP audio data, a third party gets access to the API and forwards the data stream to the system. 
This stream comes via UDP and contains 20ms of uncompressed WAV audio data for each chunk.
Each chunk carries an identifiable tag that allows the system to map the audio data against a user.
These chunks get concatenated to be processed by the speech-processing pipeline.


\chapter{Audio Data Reception}

\label{AudioDataReception}

The system is versatile and is capable of handling a wide variety of input audio data streams. The scope of this work, 
however, focuses on the reception of web browser audio input and Voice over IP audio input.

%----------------------------------------------------------------------------------------

\section{Audio Connection}

Any specific audio reception point returns the generic AudioConnection class and allows the rest of the system to 
handle every audio connection the same. 
It contains data about its assigned session, the user it belongs to, and the audio input channel it uses.

It also provides an event-based interface so the rest of the system can listen to specific events and react  
accordingly. The available events are the "close" event, which gets fired when the audio connection gets closed, and 
the "message" event, which gets fired when new data from the audio stream is available.

This architecture provides a unified interface for the entire system to handle audio streams and their data.

\begin{verbatim}
// audioConnection.ts
export type EventType = "message" | "close";
export class AudioConnection {
    get closed() {}
    readonly id: string;
    readonly sessionId: string;
    readonly userId: string;
    readonly userName: string | undefined;
    readonly rtpRegisterEntry: RTPRegisterEntry | undefined;

    constructor(
        id: string,
        sessionId: string,
        userId?: string,
        userName?: string,
        rtpRegisterEntry?: RTPRegisterEntry,
    ) {}

    emitMessage(message: Buffer) {}
    close() {}
    on(type: EventType, cb: (arg: Buffer | void) => void): number {}
    off(id: number) {}
}
\end{verbatim}


\chapter{Speech Processing}

\label{SpeechProcessing}

The speech-processing component mainly comprises a microservice wrapping OpenAIs Whisper project and the management 
that utilizes this component to process audio streams. It receives slices of the audio stream, various in length, and
transcribes them into text.

Since Whisper cannot process audio streams directly but solely complete audio files, this service is required to build 
the basis to transcribe slices of the audio stream.


\chapter{Session Handling and Message Translation}

\label{SessionHandlingAndMessageTranslation}

The session handling and message translation components are responsible for managing the interaction between the 
different session participants and translating the messages between the different languages if necessary.

%----------------------------------------------------------------------------------------

\section{Session Handling}

A session represents a single conversation between multiple participants. It contains all the information about the 
users, their audio connections, and the messages sent in this session. The session handling component is 
responsible for creating and managing these interactions.

After a session is created, it is assigned a unique session ID. This session ID is used to identify the session and 
allows the participants to join the session. After a user joins, the session registers the user and keeps track of 
their session-related data, like their determined language.

\subsection{Message Handling}

After a user object identifies the start of a new message, it creates a new message object and assigns it to the 
session. The session receives all updates regarding the message and notifies all subscribed participants about the 
changes. Users connected to a session via Web Browser can see the updates to the messages in real time since the 
session sends the updates via WebSockets.

After a message is complete and its processing is finished, the session determines the language of the message and 
finds all foreign languages present in the session by different users. It then requests all necessary translations. 

If a message transcription request takes longer than 1500ms, the following request is postponed until the previous 
request is finished. This prevents the transcription services described in Chapter \ref{SpeechProcessing} from 
overloading. This safeguard is necessary to prevent the transcription service from building up a backlog of requests 
that can not be processed in time. This would result in a continuously growing backlog and eventually increase 
processing times.

\subsection{Interface Options}

A session provides multiple options to interact with it. One of them is the WebSocket interface, which allows 
subscribing and unsubscribing to events like creating and updating messages or updating the entire session object, 
for instance, when a new user joins the session. The web browser client uses this interface to receive updates about 
the session and its messages to display them to the user.

Another option is a WebHook interface. This allows third-party applications to subscribe to the session and receive 
updates about it and its messages. If a service registers to a WebHook, a session sends all updates to the \ac{http} 
endpoint of the service. This interface can be used by any third-party application to integrate with the System and 
receive updates about the session.

\subsection{Bot Interface}

The WebHook interface can also integrate a bot into the session context. Any application can register to the WebHook 
interface and receive updates about the session. This allows the application to interact with the session and its 
participants. The application can send messages to the session and receive updates about the session and its messages. 
This allows the application to interact with the ongoing conversation. The bot can even post messages in any language. 
Due to the translation capabilities of the System, the participants can read and or listen to the message in their 
preferred language.

This feature can be used to integrate a chatbot that can analyze the conversation and respond to it accordingly. 
However, example implementations of such a bot are outside the scope of this work and will not be further discussed.

%----------------------------------------------------------------------------------------

\section{Message Translation}

The message translation component is responsible for translating messages in different languages. It uses DeepL to 
translate the message content and returns the translated message to the session handling component.

It gets triggered by the session after a message is complete and its audio processing is finished. The session 
determines the language of the message and finds all foreign languages present in the session by different users. 
It then requests all necessary translations. 

After providing the translated message for each requested language, the session notifies all subscribed parties about 
the newly available translation. The web browser client uses this to display the translated message to the user. 
Third-party applications can use the WebHook interface to receive and integrate the translated message into their 
application.

The translation is also used to synthesize the translated message in a foreign language. The session handling component 
uses the synthesized audio file to play it back to the users in a foreign language. Especially in the case of a 
\ac{voip} connection, this is the only way to provide the translated message to the participant and allows them to 
communicate with each other. However, this is described in more detail in Chapter 
\ref{SpeechSynthesisAndAudioDataTransmission}.\\

The translation service is the only component of the speech-processing pipeline that does not run on-premise. It is an 
external service DeepL provides and can not be hosted on demand. It is, however, possible to host a similar translation 
service on-premise and use it instead of DeepL. The speech-processing pipeline is designed to be versatile and allows 
the integration of any translation service with relative ease.

\subsection{Marian NMT}

As an open-source alternative to DeepL, Marian NMT (\cite{mariannmt}) is a possible candidate. It is an open-source 
translation service that can be hosted on-premise. After experimenting, it was found that the translation quality was 
not on par with DeepL. Since the System aims to provide a high-quality translation in a critical environment, it was 
decided to use DeepL instead of Marian NMT.

This makes the translation service the only component of the speech-processing pipeline that is not open-source. 
However, the System is designed to allow the integration of any translation service with little effort. This permits 
the observation of the development of open-source translation services and integration into the System if they reach 
the required quality.

\chapter{Speech Synthesis and Audio Data Transmission}

\label{SpeechSynthesisAndAudioDataTransmission}

Speech synthesis and transmitting the resulting audio data to the proper users is crucial for the System to work. The 
following components are responsible for this task. Especially for users only connected via Voice over IP, this is the 
only way to provide them with the translated message.

%----------------------------------------------------------------------------------------

\section{Speech Synthesis}

The System uses PiperTTS, an open-source text-to-speech library, to synthesize spoken audio from text. It supports 
multiple languages and voices and allows the System to create audio files in any language with a voice model.

The System uses PiperTTS to create audio files from the translated messages. The audio files are stored in the 
filesystem and available via a unique ID. The ID gets returned to the session handling component, which uses it to send 
the audio file to the appropriate users.

\subsection{TTS Microservice}

Like the transcription microservice, the speech synthesis microservice wraps the PiperTTS CLI interface and 
provides a convenient HTTP API interface to the rest of the System. The service spawns a child process that runs the 
PiperTTS CLI application and awaits the generation of the audio file. It then returns the audio file as an HTTP response 
encoded as audio/wav.

The HTTP interface of the microservice is straightforward. It provides a single endpoint for synthesizing audio files 
and returns the resulting audio file. The endpoint is available under the root path of the microservice and expects a 
POST request with the text to synthesize as the body of the request. There is one required query parameter to specify 
the language of the audio content. Additionally, there are optional query parameters to specify the preferred country, 
gender, or a specific voice model. The microservice returns the resulting audio file.

\begin{verbatim}
POST / HTTP/1.1
Content-Type: text/plain
Content-Length: 28
Accept: audio/wav

Hello World, this is a test.

\end{verbatim}

\subsection{Audio Generation}

Since PiperTTS utilizes the entire CPU by splitting the audio generation into as many threads as there are CPU cores, 
the System can utilize the hardware's full potential without creating a service pool like the transcription 
microservice.

The generated audio file is in the WAV format and encoded as audio/wav. The System stores the audio file in the 
filesystem. It returns the ID of the audio file to the session handling component—the session handling component uses 
the ID to send the audio file to the appropriate users. 

%----------------------------------------------------------------------------------------

\section{Audio Data Transmission}

After the audio file is available, the System needs to send it to the appropriate users. This procedure is different 
depending on the audio input channel. The System uses WebSockets to send the audio file to users connected via a web 
browser. For users connected via Voice over IP, the System uses a UDP backchannel to send the audio file to the VoIP 
provider.

\subsection{Web Browser Audio}

If the user connects via a browser audio stream, the System sends a WebSocket message to the client to notify it about 
the existence of a newly synthesized file. The web client uses the transmitted ID property to download the audio WAV 
file and uses the Web Audio API to play it back.

The WebSocket message contains the ID of the audio file as well as data about the nature of the synthesized audio file. 
The web client uses this information to determine the correct way to play back the audio file. The System sends the 
following WebSocket message to the client.

\begin{verbatim}
{
    "type": "tts",
    "audioId": "1c4e0c77-f3d9-5ba5-bc88-c2177926514d",
    "playbackType": "translation"
}
\end{verbatim}

The "type" property allows the use of one single WebSocket connection for all kinds of messages and helps the client 
determine the correct way to handle the message. In this case, it is a text-to-speech event. The "audioId" property 
contains the ID of the audio file. The "playbackType" property contains the type of the audio file. In this case, it is 
a translation of a message sent by another user. It can also be assigned to the value "translation-playback." The 
"translation-playback" value is used to play back the translation of the user's message.

With this design decision, the User Interface can easily support features to play back the translations in different 
volume levels or even allow client-side configuration to turn off the playback of specific translations altogether.

\subsection{Voice over IP Audio}

If the user connects via a VoIP provider, the System sends a UDP message to the provider via a previously configured 
backchannel port. The message contains the ID of the audio file, and the VoIP provider uses the ID to download the 
audio WAV file and plays it back into the audio stream of the phone call. 

The UDP message contains only the ID of the audio file. The VoIP provider uses the ID to download the audio file and 
play it back to the user. The System sends the following UDP message to the VoIP provider. 

\begin{verbatim}
WaveFile:30ea2c43-bbdc-5db6-8d46-78ddbc2b1b72
\end{verbatim}

\chapter{Evaluation}

\label{Evaluation}

This chapter evaluates the System's capabilities and compares them to the requirements in Chapter 
\ref{Introduction}. It analyzes the System's performance and its ability to handle multiple audio streams 
simultaneously.

%----------------------------------------------------------------------------------------

\section{Streaming Capability \& Transcription Interval}

Since Whisper does not support real-time audio transcription, the System had to be designed to handle audio streams 
and utilize Whisper to transcribe slices of the audio stream. By implementing this feature, the System 
enhances the capabilities of Whisper and allows it to transcribe audio streams in near real-time.

The System slices the stream into chunks and transcribes them individually. The frequency of the transcription - 
and therefore the size of the chunks - is configurable using the "TranscriptionInterval" environment variable.

The following analysis compares the System's required transcription processing time with various amounts of 
simultaneously processed audio streams. The System runs on a machine with two Nvidia GeForce RTX 4070 graphics cards. 

The following table and graphic show the average processing times for one, two, four, and eight simultaneous audio 
streams. Each is the average processing time of 100 iterations of 30-second audio streams.

\begin{figure}[ht]
	\includegraphics[width=\textwidth]{Figures/avg-processing-times.png}
	\caption{Average Processing Times Plot}
	\label{fig:avgProcessingTimesPlot}
\end{figure}

\begin{table}[ht]
	\begin{center}
		\begin{tabular}{ |c|c| } 
			\hline
			number of simultanious audio-stream processes & average processing time \\
			\hline\hline
			1x simultanious audio-stream & 519ms \\
			\hline
			2x simultanious audio-stream & 575ms \\
			\hline
			4x simultanious audio-stream & 983ms \\
			\hline
			8x simultanious audio-stream & 1,693ms \\
			\hline
		\end{tabular}
	\end{center}
	\caption{Average Processing Times Table}
	\label{tab:avgProcessingTimesTable}
\end{table}

\pagebreak

The diagram shows that the average processing time for one audio stream is 519 milliseconds. This would suggest that a 
transcription interval of about 520 milliseconds would be optimal for a hardware configuration like this. 

However, the diagram also reveals that the average processing time for two audio streams is 575 milliseconds. This 
seems unintuitive at first glance, but it is because the System utilizes a hardware setup with two graphics cards. 
The System can process two audio streams simultaneously, each on a separate graphics card. This results in a minor 
processing time increase for each audio stream since the System only shares CPU and data transmission resources between 
the two audio streams - not the GPU performing the transcription.

Looking at the average processing time for four simultaneous audio streams, we can see that it is 983 milliseconds. 
This significantly increases compared to the average processing time for two audio streams. This is because the System 
has to share the GPU resources between four audio streams - two on each graphics card. This results in a significant 
increase in processing time for each audio stream.

As shown in the last row of the table above, the average processing time for eight simultaneous audio streams is 
1,693 milliseconds. This significantly increases compared to the average processing time for four audio streams. 
This is because the System has to share the GPU resources between eight audio streams - four on each graphics card. 

\subsection{GPU Utilization}

The observation that the processing time increases are not doubled when the number of audio streams is doubled suggests 
that the utilization of the GPU is improved by running multiple audio transcriptions simultaneously. The GPU is only 
partially utilized when processing a single audio stream. A second, third, or fourth audio stream can utilize more GPU 
resources even though the processing time increases overall.

\subsection{1500ms Transcription Interval}

To ensure the System can handle multiple audio streams simultaneously, it uses a transcription interval of 1500 
milliseconds by default. This transcription interval is configurable using the "TranscriptionInterval" environment 
variable. The transcription interval is the time between two transcription requests of the same audio stream.

1500 milliseconds is a good default value for the transcription interval because it allows the System to support up to 
eight simultaneous audio streams on a machine with two Nvidia GeForce RTX 4070 graphics cards. When all eight audio 
streams are active, the System has a processing time greater than the transcription interval. To prevent this, 
the System awaits the results of the previous transcription before starting a new one for the same audio stream. This 
ensures that the System can handle up to eight simultaneous audio streams without requesting more resources than are 
available.

%----------------------------------------------------------------------------------------

\section{Basesd on Open Source}

The goal was to create a System based on open-source components and bundle them into a single pipeline that provides a 
complete solution for a near real-time speech processing application. The System uses open-source components for the 
most part. Audio transcription is based on the open-source solution of OpenAI called Whisper, and Speech Synthesis 
utilizes the open-source text-to-speech library PiperTTS.

The only component that is not open-source is the text translation component. Since open-source translation 
services do not meet the expectations in quality and or versatility, the design decision was made to use the DeepL 
translation service. DeepL is a commercial translation service that provides a high-quality translation API. However, 
the versatility of the system design allows for a future replacement of the DeepL translation service with an 
open-source solution with minimal effort.

\chapter{Conclusion}

\label{Conclusion}

This chapter summarizes the System's capabilities and compares them to the requirements defined in Chapter 
\ref{Introduction}. It also discusses the System's limitations and potential future work.



\end{document}  
