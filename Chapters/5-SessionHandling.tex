\chapter{Session Handling and Message Translation}

\label{SessionHandlingAndMessageTranslation}

The session handling and message translation components are responsible for managing the interaction between the 
different session participants and translating the messages between the different languages if necessary.

%----------------------------------------------------------------------------------------

\section{Session Handling}

A session represents a single conversation between multiple participants. It contains all the information about the 
users, their audio connections, and the messages sent in this session. The session handling component is 
responsible for creating and managing these interactions.

After a session is created, it is assigned a unique session ID. This session ID is used to identify the session and 
allows the participants to join the session. After a user joins, the session registers the user and keeps track of 
their session-related data, like their determined language.

\subsection{Message Handling}

After a user object identifies the start of a new message, it creates a new message object and assigns it to the 
session. The session receives all updates regarding the message and notifies all subscribed participants about the 
changes. Users connected to a session via Web Browser can see the updates to the messages in real time since the 
session sends the updates via WebSockets.

After a message is complete and its processing is finished, the session determines the language of the message and 
finds all foreign languages present in the session by different users. It then requests all necessary translations. 

If a message transcription request takes longer than 1500ms, the following request is postponed until the previous 
request is finished. This prevents the transcription services described in Chapter \ref{SpeechProcessing} from 
overloading. This safeguard is necessary to prevent the transcription service from building up a backlog of requests 
that can not be processed in time. This would result in a continuously growing backlog and eventually increase 
processing times.

\subsection{Interface Options}

A session provides multiple options to interact with it. One of them is the WebSocket interface, which allows 
subscribing and unsubscribing to events like creating and updating messages or updating the entire session object, 
for instance, when a new user joins the session. The web browser client uses this interface to receive updates about 
the session and its messages to display them to the user.

Another option is a WebHook interface. This allows third-party applications to subscribe to the session and receive 
updates about it and its messages. If a service registers to a WebHook, a session sends all updates to the \ac{http} 
endpoint of the service. This interface can be used by any third-party application to integrate with the System and 
receive updates about the session.

\subsection{Bot Interface}

The WebHook interface can also integrate a bot into the session context. Any application can register to the WebHook 
interface and receive updates about the session. This allows the application to interact with the session and its 
participants. The application can send messages to the session and receive updates about the session and its messages. 
This allows the application to interact with the ongoing conversation. The bot can even post messages in any language. 
Due to the translation capabilities of the System, the participants can read and or listen to the message in their 
preferred language.

This feature can be used to integrate a chatbot that can analyze the conversation and respond to it accordingly. 
However, example implementations of such a bot are outside the scope of this work and will not be further discussed.

%----------------------------------------------------------------------------------------

\section{Message Translation}

The message translation component is responsible for translating messages in different languages. It uses DeepL to 
translate the message content and returns the translated message to the session handling component.

It gets triggered by the session after a message is complete and its audio processing is finished. The session 
determines the language of the message and finds all foreign languages present in the session by different users. 
It then requests all necessary translations. 

After providing the translated message for each requested language, the session notifies all subscribed parties about 
the newly available translation. The web browser client uses this to display the translated message to the user. 
Third-party applications can use the WebHook interface to receive and integrate the translated message into their 
application.

The translation is also used to synthesize the translated message in a foreign language. The session handling component 
uses the synthesized audio file to play it back to the users in a foreign language. Especially in the case of a 
\ac{voip} connection, this is the only way to provide the translated message to the participant and allows them to 
communicate with each other. However, this is described in more detail in Chapter 
\ref{SpeechSynthesisAndAudioDataTransmission}.\\

The translation service is the only component of the speech-processing pipeline that does not run on-premise. It is an 
external service DeepL provides and can not be hosted on demand. It is, however, possible to host a similar translation 
service on-premise and use it instead of DeepL. The speech-processing pipeline is designed to be versatile and allows 
the integration of any translation service with relative ease.

\subsection{Marian NMT}

As an open-source alternative to DeepL, Marian NMT (\cite{mariannmt}) is a possible candidate. It is an open-source 
translation service that can be hosted on-premise. After experimenting, it was found that the translation quality was 
not on par with DeepL. Since the System aims to provide a high-quality translation in a critical environment, it was 
decided to use DeepL instead of Marian NMT.

This makes the translation service the only component of the speech-processing pipeline that is not open-source. 
However, the System is designed to allow the integration of any translation service with little effort. This permits 
the observation of the development of open-source translation services and integration into the System if they reach 
the required quality.
