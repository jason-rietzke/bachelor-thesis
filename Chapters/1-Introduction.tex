\chapter{Introduction}

\label{Chapter1}

%----------------------------------------------------------------------------------------

% Define some commands to keep the formatting separated from the content 
\newcommand{\keyword}[1]{\textbf{#1}}
\newcommand{\tabhead}[1]{\textbf{#1}}
\newcommand{\code}[1]{\texttt{#1}}
\newcommand{\file}[1]{\texttt{\bfseries#1}}
\newcommand{\option}[1]{\texttt{\itshape#1}}

%----------------------------------------------------------------------------------------

\section{Background \& Motivation}

% This work is conducted in the context of the Notitia Suite of the LiveReader GmbH, within the SPELL research project. 
% The SPELL research projects aims to improve the software of critical infrastructures, such as emergency control 
% centers. In this context the Notitia Suite serves the prupose of a interface for the emergency calltaker to dispatch 
% the emergency services. \\
% \\
% An important part of dispatching the proper resources is to understand the needs of the emergency caller. 
% This is especially difficult when the caller and the calltaker do not speak the same language and important details 
% can not be conveyed properly. \\
% \\
% This work aims to develop a speech-processing pipeline that can be integrated into the Notitia Suite to allow for 
% communication between the caller and the calltaker speaking foreign languages. The pipeline itself is able to work 
% stand alone and can be integrated into other applications but does not have to be. \\
% This work will focus on the development of the speech-processing pipeline from here on and won't go into detail 
% regarding intigration into the Notitia Suite or any other application that may use it as a service.

Effective communication is critical in emergency services, where timely and accurate information is vital to 
saving lives. The Notitia application, within the context of the SPELL research project, aims to establish a 
communication framework between emergency control centers and callers. The project envisions a seamless integration of 
a speech-processing pipeline to transcribe and translate audio streams in real-time, addressing the language barriers 
that often impede clear communication during emergencies.

The SPELL research project recognizes the importance of near real-time transcription and translation services in empowering emergency control centers—existing technologies present challenges that need innovative solutions. 
Many systems designed for continuous audio streams are cloud-based, introducing concerns regarding reliability, 
latency, and request priority—critical factors in emergency services. The proposed solution aims to 
overcome these challenges by leveraging open-source technologies and building a testable prototype that can be deployed 
on-premise, ensuring reliability in time-sensitive scenarios.

%----------------------------------------------------------------------------------------

\section{Problem Statement}

% Despite the advancements in speech processing, existing solutions for transcription and translation in continuous audio 
% streams often fall short of meeting the specific demands of emergency call centers. Cloud-based systems, while widely 
% available, pose risks related to their reliability and latency, which are critical considerations in the context of 
% emergency response. The inability to seamlessly integrate features from different providers further hinders the 
% optimization of the transcription and translation process.\\
% \\
% The challenge, therefore, is to develop a speech-processing pipeline that not only addresses the language barriers in 
% emergency communication but also meets the stringent requirements of reliability, low latency, and on-premise 
% deployment. The integration of OpenAI Whisper, DeepL, and PiperTTS as foundational technologies sets the stage for a 
% comprehensive solution, yet the task is to weave these components into a coherent system tailored for emergency 
% services.

Despite the rapid advancements in speech processing technologies, the unique demands of emergency call centers pose 
significant challenges that current solutions need help to overcome. The SPELL research project identifies a crucial gap 
in systems designed for continuous audio streams. While cloud-based platforms offer convenience and 
scalability, they introduce a host of concerns when applied to the critical infrastructure of emergency services. 
Issues such as reliability, latency, and request prioritization become magnified in time-sensitive scenarios, 
jeopardizing the effectiveness of emergency responses.

The reliance on cloud solutions also restricts the flexibility to deploy services on-premise, a key consideration for 
maintaining the operational integrity of emergency control centers. Moreover, the homogeneous nature of many 
speech-processing systems impedes the seamless integration of features from diverse providers. This limitation prevents 
emergency services from harnessing the full spectrum of capabilities offered by various tools, hindering the 
optimization of transcription and translation processes.

In essence, the challenge is not just about overcoming language barriers; it is about tailoring a solution that aligns 
with the unique requirements of emergency communication. The proposed speech-processing pipeline must be a fusion of 
innovation and practicality, addressing linguistic diversity and the critical need for reliability, low 
latency, and the ability to operate on-premise.

Therefore, the problem is multifaceted: How can a speech processing pipeline be meticulously crafted to 
transcend the limitations of current technologies, providing a reliable, low-latency, and on-premise solution for 
emergency call centers? This challenge necessitates an exploration of open-source technologies, strategic integration 
of proven components like OpenAI Whisper and DeepL, and the development of a tailored prototype, all within the 
stringent time constraints that characterize emergencies. The task is to navigate the intricacies of emergency 
communication, ensuring that the proposed solution becomes an indispensable tool for effective and timely response in 
the face of linguistic diversity.


%----------------------------------------------------------------------------------------

\section{Objectives \& Scope}

The primary objective of this bachelor's thesis is to develop a testable prototype for a speech-processing pipeline 
within the Notitia application. This prototype aims to transcribe and translate audio streams in near real-time, 
facilitating communication between emergency control centers and callers speaking different languages. The focus is on 
utilizing existing open-source technologies, such as OpenAI Whisper for speech recognition, DeepL for translation, and 
PiperTTS for speech synthesis.

The scope encompasses the integration of the speech-processing pipeline into the Notitia application, leveraging 
existing internal modules for user authentication and data-stream integration to the Voice over IP provider. The 
system's functionality revolves around processing audio streams, segmenting them into individual messages, and managing 
them within a session context specific to foreign language communication.

%----------------------------------------------------------------------------------------

\section{Research Questions \& Methodology}

\subsection{Research Questions}

How can a speech processing pipeline, predominantly built upon open-source software, be harnessed to facilitate near 
real-time transcription and translation of spoken content to overcome language barriers in the context of an emergency 
call center?

\subsection{Methodology}

The research methodology involves the development of a testable prototype with a focus on minimizing processing time. 
The integration of OpenAI Whisper, DeepL, and PiperTTS will be explored to optimize the transcription and translation 
process. The system will utilize existing internal modules for user authentication and data-stream integration, with a 
data stream transferring audio in WAV format. The research emphasizes on-premise deployment for reliability in 
emergency scenarios.

In the subsequent chapters, we delve into the technical details of the speech-processing pipeline, its integration into 
Notitia, and the evaluation of its performance in the critical context of emergency communication.

