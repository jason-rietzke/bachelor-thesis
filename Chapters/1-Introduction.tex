\chapter{Introduction}

\label{Introduction}

%----------------------------------------------------------------------------------------

\section{Background \& Motivation}

Effective communication is critical in emergency services, where timely and accurate information is vital to 
saving lives. The Notitia (\cite{notitia2023}) application of LiveReader, within the context of the 
\cite{spell2023} research project, aims to establish a communication framework between 
emergency control centers and callers. The project envisions a seamless integration of a speech-processing pipeline to 
transcribe and translate audio streams in real-time, addressing the language barriers that often impede clear 
communication during emergencies.

The SPELL research project recognizes the importance of near real-time transcription and translation services in 
empowering emergency control centers — existing technologies present challenges that need innovative solutions. 
Many systems designed for continuous audio streams are cloud-based, introducing concerns regarding reliability, 
latency, and request priority — critical factors in emergency services. The proposed solution aims to 
overcome these challenges by leveraging open-source technologies and building a testable prototype that can be deployed 
on-premise, ensuring reliability in time-sensitive scenarios.

\subsection{Notitia \& SPELL}

Notitia is a web-based application built by LiveReader designed to design knowledge graphs and use them to run business 
processes and workflows supported by symbolic AI. As part of the SPELL research project, Notitia is being used to 
integrate the expert knowledge of emergency control centers, allow the call takers to easily collect data about the 
emergency, and provide the caller with instructions on how to proceed.

Throughout the SPELL research project, the necessity of a speech processing pipeline became apparent. The pipeline 
should be able to transcribe and translate audio streams in near real-time. This would allow emergency control centers 
to communicate with callers in foreign languages without requiring a human translator.

Future internal work at LiveReader will focus on integrating the speech processing pipeline into Notitia to extend its 
capabilities and provide a language understanding component. This will allow Notitia to understand the spoken content 
without requiring a human to enter all the information manually. This aspect, however, is outside the scope of this 
thesis.

%----------------------------------------------------------------------------------------

\section{Problem Statement}

Despite the rapid advancements in speech processing technologies, the unique demands of emergency call centers pose 
significant challenges that current solutions need help to overcome. The SPELL research project identifies a crucial 
gap in systems designed for continuous audio streams. While cloud-based platforms offer convenience and 
scalability, they introduce a host of concerns when applied to the critical infrastructure of emergency services. 
Issues such as reliability, latency, and request prioritization become magnified in time-sensitive scenarios, 
jeopardizing the effectiveness of emergency responses.

The reliance on cloud solutions also restricts the flexibility to deploy services on-premise, a key consideration for 
maintaining the operational integrity of emergency control centers. Moreover, the homogeneous nature of many 
speech-processing systems impedes the seamless integration of features from diverse providers. This limitation prevents 
emergency services from harnessing the full spectrum of capabilities offered by various tools, hindering the 
optimization of transcription and translation processes.

In essence, the challenge is not just about overcoming language barriers; it is about tailoring a solution that aligns 
with the unique requirements of emergency communication. The proposed speech-processing pipeline must be a fusion of 
innovation and practicality, addressing linguistic diversity and the critical need for reliability, low 
latency, and the ability to operate on-premise.

%----------------------------------------------------------------------------------------

\section{Objectives \& Scope}

The primary objective of this bachelor's thesis is to develop a testable prototype for a speech-processing pipeline 
within the context of the Notitia application. This prototype aims to transcribe and translate audio streams in near 
real-time, facilitating communication between emergency control centers and callers speaking different languages. The 
focus is on utilizing open-source technologies, such as OpenAI Whisper for speech recognition and PiperTTS for 
speech synthesis.

The scope encompasses the integration of the speech-processing pipeline into the Notitia application, leveraging 
existing internal modules for user authentication and data-stream integration to the \ac{voip} provider. The 
system's functionality revolves around processing audio streams, segmenting them into individual messages, and managing 
them within a session context specific to foreign language communication.

%----------------------------------------------------------------------------------------

\section{Research Questions \& Methodology}

\subsection{Research Questions}

How can a speech processing pipeline, predominantly built upon open-source software, be harnessed to facilitate near 
real-time transcription and translation of spoken content to overcome language barriers in the context of an emergency 
call center?

\subsection{Methodology}

The research methodology involves the development of a testable prototype with a focus on minimizing processing time. 
The integration of OpenAI Whisper will be explored to optimize the transcription process. The System will utilize 
existing internal modules for user authentication and data-stream integration, with a data stream transferring audio 
in \ac{wav} format. The research emphasizes finding a sufficient processing time for the transcription of audio streams.

In the subsequent chapters, we delve into the technical details of the speech-processing pipeline and evaluate its 
performance.

%----------------------------------------------------------------------------------------

\section{Server Hardware}

The System is designed to run on commodity hardware. LiveReader provides a server setup with two Nvidia GeForce RTX 
4070 graphics cards. The System utilizes the \ac{gpu} for audio transcription and speech synthesis. The System is 
designed to run on this hardware configuration, and all measurements are based on this setup.

%----------------------------------------------------------------------------------------

\section{Development Process}

For the development of this project, Visual Studio Code is the \ac{ide} of choice. The project utilizes a microservice 
architecture, which allows the development of different system parts independently and deploying them separately. 
The project is versioned using Git and hosted privately on GitHub under the LiveReader organization. 

The primary runtime for the backend is Node.js version 20 LTS. It provides an \ac{api} for any frontend or third party 
to communicate with. The web interface uses Vue.js as a UI framework, extended by Vuetify as a UI component library.
