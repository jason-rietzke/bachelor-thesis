\chapter{Speech Synthesis and Audio Data Transmission}

\label{SpeechSynthesisAndAudioDataTransmission}

Speech synthesis and transmitting the resulting audio data to the proper users is crucial for the System to work. The 
following components are responsible for this task. Especially for users only connected via \ac{voip}, this is the 
only way to provide them with the translated message.

%----------------------------------------------------------------------------------------

\section{Speech Synthesis}

The System uses PiperTTS (\cite{rhasspy2023pipertts}), an open-source text-to-speech library, to synthesize spoken 
audio from text. It supports multiple languages and voices and allows the System to create audio files in any language 
with a voice model.

The System uses PiperTTS to create audio files from the translated messages. The audio files are stored in the 
filesystem and available via a unique ID. The ID gets returned to the session handling component, which uses it to send 
the audio file to the appropriate users.

\subsection{TTS Microservice}

Like the transcription microservice, the speech synthesis microservice wraps the PiperTTS \ac{cli} and provides a 
convenient \ac{http} \ac{api} to the rest of the System. The service spawns a child process that runs the 
PiperTTS \ac{cli} application and awaits the generation of the audio file. It then returns the audio file as an 
\ac{http} response encoded as audio/wav.

The \ac{http} interface of the microservice is straightforward. It provides a single endpoint for synthesizing audio 
files and returns the resulting audio file. The endpoint is available under the root path of the microservice and 
expects a POST request with the text to synthesize as the body of the request. There is one required query parameter to 
specify the language of the audio content. Additionally, there are optional query parameters to specify the preferred 
country, gender, or a specific voice model. The microservice returns the resulting audio file.

\begin{verbatim}
POST / HTTP/1.1
Content-Type: text/plain
Content-Length: 28
Accept: audio/wav

Hello World, this is a test.

\end{verbatim}

\subsection{Audio Generation}

Since PiperTTS utilizes the entire \ac{cpu} by splitting the audio generation into as many threads as there are 
\ac{cpu} cores, the System can utilize the hardware's full potential without creating a service pool like the 
transcription microservice.

PiperTTS also supports \ac{cuda} acceleration, which allows the System to utilize the \ac{gpu} for audio generation. 
However, this did not significantly improve the performance of the generation. Since the \ac{gpu} is also used for 
transcription, the System does not use the \ac{gpu} acceleration for audio generation.

The generated audio file is in the \ac{wav} format and encoded as audio/wav. The System stores the audio file in the 
filesystem. It returns the ID of the audio file to the session handling component—the session handling component uses 
the ID to send the audio file to the appropriate users. 

%----------------------------------------------------------------------------------------

\section{Audio Data Transmission}

After the audio file is available, the System needs to send it to the appropriate users. This procedure is different 
depending on the audio input channel. The System uses WebSockets to send the audio file to users connected via a web 
browser. For users connected via \ac{voip}, the System uses a \ac{udp} backchannel to send the audio file to the 
\ac{voip} provider.

\subsection{Web Browser Audio}

If the user connects via a browser audio stream, the System sends a WebSocket message to the client to notify it about 
the existence of a newly synthesized file. The web client uses the transmitted ID property to download the audio 
\ac{wav} file and uses the Web Audio \ac{api} to play it back.

The WebSocket message contains the ID of the audio file as well as data about the nature of the synthesized audio file. 
The web client uses this information to determine the correct way to play back the audio file. The System sends the 
following WebSocket message to the client.

\begin{verbatim}
{
    "type": "tts",
    "audioId": "1c4e0c77-f3d9-5ba5-bc88-c2177926514d",
    "playbackType": "translation"
}
\end{verbatim}

The "type" property allows the use of one single WebSocket connection for all kinds of messages and helps the client 
determine the correct way to handle the message. In this case, it is a text-to-speech event. The "audioId" property 
contains the ID of the audio file. The "playbackType" property contains the type of the audio file. In this case, it is 
a translation of a message sent by another user. It can also be assigned to the value "translation-playback." The 
"translation-playback" value is used to play back the translation of the user's message.

With this design decision, the User Interface can easily support features to play back the translations in different 
volume levels or even allow client-side configuration to turn off the playback of specific synthesized messages 
altogether.

\subsection{Voice over IP Audio}

If the user connects via a \ac{voip} provider, the System sends a \ac{udp} message to the provider via a previously 
configured backchannel port. The message contains the ID of the audio file, and the \ac{voip} provider uses the ID to 
download the audio \ac{wav} file and plays it back into the audio stream of the phone call.

The System sends the following \ac{udp} message to the \ac{voip} provider. 

\begin{verbatim}
WaveFile:30ea2c43-bbdc-5db6-8d46-78ddbc2b1b72
\end{verbatim}

This gives the \ac{voip} provider the flexibility to play back the audio file in any way it wants. It can also 
implement features to play back the audio file at different volume levels. One possible implementation would be to 
playback the translation in total volume and lower the volume of the original audio stream. This would allow the user 
to hear the translation more clearly and still have the original audio stream in the background.

\section{Audio Download}

For the web client or the \ac{voip} provider to download the audio file, the System provides a simple \ac{http} 
endpoint. The endpoint is available under the /api/v1/audio path of the speech-processing pipeline. It expects a GET 
request and requires the ID of the audio file as a query parameter. The System returns the audio file as an 
audio/wav response.

\begin{verbatim}
HTTP/1.1 200 OK
Content-Type: audio/wav
Content-Length: 1234567

RIFF....WAVEfmt ........data
\end{verbatim}

This simple \ac{http} \ac{api} makes it easy for all components of the System and third parties to download the audio 
file and integrate it into their respective applications.
