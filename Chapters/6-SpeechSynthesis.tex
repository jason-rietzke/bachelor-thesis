\chapter{Speech Synthesis and Audio Data Transmission}

\label{SpeechSynthesisAndAudioDataTransmission}

Speech synthesis and transmitting the resulting audio data to the proper users is crucial for the System to work. The 
following components are responsible for this task. Especially for users only connected via Voice over IP, this is the 
only way to provide them with the translated message.

%----------------------------------------------------------------------------------------

\section{Speech Synthesis}

The System uses PiperTTS, an open-source text-to-speech library, to synthesize spoken audio from text. It supports 
multiple languages and voices and allows the System to create audio files in any language with a voice model.

The System uses PiperTTS to create audio files from the translated messages. The audio files are stored in the 
filesystem and available via a unique ID. The ID gets returned to the session handling component, which uses it to send 
the audio file to the appropriate users.

\subsection{TTS Microservice}

Like the transcription microservice, the speech synthesis microservice wraps the PiperTTS CLI interface and 
provides a convenient HTTP API interface to the rest of the System. The service spawns a child process that runs the 
PiperTTS CLI application and awaits the generation of the audio file. It then returns the audio file as an HTTP response 
encoded as audio/wav.

The HTTP interface of the microservice is straightforward. It provides a single endpoint for synthesizing audio files 
and returns the resulting audio file. The endpoint is available under the root path of the microservice and expects a 
POST request with the text to synthesize as the body of the request. There is one required query parameter to specify 
the language of the audio content. Additionally, there are optional query parameters to specify the preferred country, 
gender, or a specific voice model. The microservice returns the resulting audio file.

\begin{verbatim}
POST / HTTP/1.1
Content-Type: text/plain
Content-Length: 28
Accept: audio/wav

Hello World, this is a test.

\end{verbatim}

\subsection{Audio Generation}

Since PiperTTS utilizes the entire CPU by splitting the audio generation into as many threads as there are CPU cores, 
the System can utilize the hardware's full potential without creating a service pool like the transcription 
microservice.

The generated audio file is in the WAV format and encoded as audio/wav. The System stores the audio file in the 
filesystem. It returns the ID of the audio file to the session handling component—the session handling component uses 
the ID to send the audio file to the appropriate users. 

