\chapter{Concept}

\label{Chapter2}

%----------------------------------------------------------------------------------------

\section{Audio Input}

The flexibility of the system allows for various audio input options. This work will cover two of the most important 
ones. The two most critical audio sources are recordings by web browsers and audio input provided by voice-over-IP 
communication systems.

\subsection{Web Browser Audio}

Within the context of this work, the focus lies on integrating Chromium-based web browsers. The Web MediaRecorder API 
provides the audio input. The API requests access to the microphone and allows for recording audio streams. 
The audio stream gets sent to the server as WebM-encoded data chunks of about 200ms. The server then concatenates the 
chunks and converts the audio stream to WAV format. The conversion utilizes FFmpeg, a command-line tool for 
manipulating audio and video files. The speech-processing pipeline then processes the resulting WAV file.

\subsection{Voice-over-IP Audio}

To get access to VoIP audio data, a third party gets access to the API and forwards the data stream to the system. 
This stream comes via UDP and contains 20ms of uncompressed WAV audio data for each chunk.
Each chunk carries an identifiable tag that allows the system to map the audio data against a user.
These chunks get concatenated to be processed by the speech-processing pipeline.

