\chapter{Concept}

\label{Chapter2}

The concept chapter of this work gives a brief overview of what components are doing and how they are related. 
A detailed look at each component and its inner workings comes in the following chapters of this work.


%----------------------------------------------------------------------------------------

\section{Audio Input}

The flexibility of the system allows for various audio input options. This work will cover two of the most important 
ones. The two most critical audio sources are recordings by web browsers and audio input provided by voice-over-IP 
communication systems.

\subsection{Web Browser Audio}

Within the context of this work, the focus lies on integrating Chromium-based web browsers. The Web MediaRecorder API 
provides the audio input. The API requests access to the microphone and allows for recording audio streams. 
The audio stream gets sent to the server as WebM-encoded data chunks of about 200ms. The server then concatenates the 
chunks and converts the audio stream to WAV format. The conversion utilizes FFmpeg, a command-line tool for 
manipulating audio and video files. The speech-processing pipeline then processes the resulting WAV file.

\subsection{Voice-over-IP Audio}

To get access to VoIP audio data, a third party gets access to the API and forwards the data stream to the system. 
This stream comes via UDP and contains 20ms of uncompressed WAV audio data for each chunk.
Each chunk carries an identifiable tag that allows the system to map the audio data against a user.
These chunks get concatenated to be processed by the speech-processing pipeline.

%----------------------------------------------------------------------------------------

\section{Speech Processing}

The speech processing component mainly comprises a microservice wrapping OpenAIs Whisper project. It receives slices of 
the audio stream, various in length, and transcribes them into text. The transcription is then analyzed, and if there 
is no change in content for more than two iterations, it finalizes the pending message. Any new content will cause the 
creation of a new message object, and the process will start over.

The speech-processing pipeline utilizes Whisper to determine the spoken language as well. It returns the most likely 
language used in the audio file. Alongside the integrated voice-activation feature to filter out sections that do not 
contain spoken content, Whisper is a very versatile tool.

After a message is determined to have ended, the system utilizes DeepL to translate the message into any foreign 
language spoken within the same session.

%----------------------------------------------------------------------------------------

\section{Audio Output}

After the translated text is available, the system can synthesize an audio file. The audio file gets created 
using PiperTTS, an open-source text-to-speech library. The library supports multiple languages and voices. The 
resulting audio file gets sent back to users via the appropriate audio output channel. This procedure is different 
depending on the audio input channel.

\subsection{Web Browser Audio}

If the user connects via a browser audio stream, the system sends a WebSocket message to the client to notify it about 
the existence of a newly synthesized file. The web client uses the transmitted ID property to download the audio WAV 
file and uses the Web Audio API to play it back.

\subsection{Voice-over-IP Audio}
If the user connects via a VoIP provider, the system sends a UDP message to the provider via a previously configured 
backchannel port. The message contains the ID of the audio file, and the VoIP provider uses the ID to download the 
audio WAV file and plays it back into the audio stream of the phone call.

%----------------------------------------------------------------------------------------

\section{Session Handling}

After creating a new session with its corresponding ID, users can join it by connecting to it with an audio stream. 
Each audio stream is related to a known Notitia User account or gets a temporary user object. In the latter case, the 
user will not be able to be identified as the same one if the audio connection drops and gets reconnected.

The session object holds various data about the connected audio streams, their users, and the determined spoken 
languages of each one.
This data leads to generating the necessary translations of a message.
A session is also responsible for broadcasting the information about generated audio files to the appropriate receivers.

It is possible to connect any number of users to one singular session, but within the context of this work, a two-user 
session is most common since it represents the use case of an emergency caller and call-taker.
