\documentclass[a4paper,10pt]{article}
\usepackage{geometry}
\usepackage{xcolor}
\usepackage{graphicx}
\usepackage{setspace}
\usepackage{titlesec}
\usepackage{hyperref}
\usepackage{cite}
\usepackage{pdfpages}

\usepackage[utf8]{inputenc}
\usepackage[T1]{fontenc}
\usepackage{stix}

\definecolor{titlepagecolor}{cmyk}{1,.60,0,.40}
\definecolor{namecolor}{cmyk}{1,.50,0,.10}

\title{Bachelor thesis Exposé: Speech-Processing-Pipeline to overcome language barriers in real-time communication}
\author{Jason Rietzke}

\begin{document}

\begin{titlepage}
\raggedleft
\rule{1pt}{\textheight}
\hspace{0.05\textwidth}
\parbox[b]{0.85\textwidth}{
  {\textbf{November 2nd 2023}}\\\\
  {\Huge\bfseries Bachelor thesis Exposé \\[0.5\baselineskip]}\\[2\baselineskip]
  {\Large\bfseries Speech-Processing-Pipeline to overcome language barriers in real-time communication \\[0.5\baselineskip]}\\[2\baselineskip]
  {\large\textit{In cooperation with the LiveReader GmbH}}\\[4\baselineskip]

  \vspace{0.28\textheight}

  {\textbf{Environmental Informatics and Business Information Systems}\\}
  {9. Semester, WS2023/2024}\\
  {Campusallee}\\
  {55768 Hoppstädten-Weiersbach}\\
  {Trier University of Applied Sciences, Environmental Campus Birkenfeld}\\
  {UPUT}\\
  \\
  {\textbf{Jason Rietzke}\\}
  \\
  {\textbf{Supervisor:} Prof. Dr. Rumpler}\\
}

\end{titlepage}

\restoregeometry
\nopagecolor


\section{Description}
This bachelor's thesis aims to develop a speech-processing pipeline primarily focusing on integrating it within the Notitia application.
It serves as a communication framework between emergency control centers and callers within the scope of the SPELL research project.
The goal is to develop a testable prototype to transcribe and translate audio streams of at least two spoken languages in one session.

The thesis seeks to compose a solution that enables near real-time transcription and translation of spoken content by utilizing existing partial solutions like OpenAI Whisper and DeepL as the foundational technologies while bridging the solutions to work appropriately as one pipeline.
OpenAI Whisper, for instance, is not meant to be used on audio streams but is open source, can be run on-premise and supports many languages with high quality out of the box.

Additionally, the system builds on top of existing internal modules for Notitia-User-Authentication and a data-stream integration to the Voice over IP provider, which is currently in preparation. This data stream will transfer audio in a WAV format.

The system's core functionality involves processing audio streams, segmenting them into individual messages, and handling them within a session context regarding the participants speaking a foreign language.


\section{Relevance}
Effective communication in emergency services is essential, especially in time-sensitive situations where immediate, accurate information can save lives.
The ability to bridge language barriers by offering near real-time transcription and translation services empowers emergency control centers to understand better and respond to callers' needs, providing practical assistance.

Existing solutions by AWS or Microsoft Azure are often only available as cloud services rather than based on open-source technologies, which makes it difficult to rely on in critical situations.


\section{Research Question}
How can a speech processing pipeline, built upon mostly open-source software, be harnessed to facilitate near real-time transcription and translation of spoken content, focusing on enhancing communication in emergency services between control centers and callers, especially when language diversity poses a significant challenge?


\section{Approach}
\begin{enumerate}
  \item Receiving audio: Receive audio streams from voice-over-IP providers.
  \item Voice activation: Recognize spoken content to begin and end separate messages from a continuous audio stream (possibly built into Whisper).
  \item Language detection: Detect the spoken language  (possibly built into Whisper).
  \item Transcription: Using OpenAI Whisper to transcribe the audio data into text.
  \item Session handling: Organize related users and their respective audio streams and transcribe messages into sessions.
  \item Translation: Check for multiple spoken languages within one session and translate the transcription if necessary.
  \item Speech synthesis: Synthesize an audio file from the transcription and feed it back into the audio stream.
\end{enumerate}


\section{Preliminary Literature and Resources}
\begin{enumerate}
  \item OpenAI Whisper Repository \url{https://github.com/openai/whisper}
  \item OpenAI Whisper Paper \url{https://arxiv.org/abs/2212.04356}
  \item FFmpeg \url{https://ffmpeg.org/}
  \item Voice Activity Detection \url{https://speechprocessingbook.aalto.fi/Recognition/Voice_activity_detection.html}
  \item AT\&T Labs Research research paper \url{https://aclanthology.org/N12-1048.pdf}
\end{enumerate}


\section{Outline}
\begin{enumerate}
  \item Introduction and motivation
  \item Concept
  \item Audio data reception and speech processing
  \subitem Voice activation
  \subitem Language detection
  \subitem Transcription
  \item Session handling and message translation
  \item Speech synthesis and audio data transmission
  \item Evaluation
  \item Conclusion
  \item Literature
\end{enumerate}


\section{Schedule}
The bachelor thesis covers nine weeks, starting on November 10, 2023, and ending on January 12, 2024.
\begin{description}
  \item[2023 CW 46] Integrate the VoIP provider as well as voice activation and language detection
  \item[2023 CW 47] Implement segmented audio transcription with OpenAI Whisper
  \item[2023 CW 48] Embed translation and speech synthesis services and build session handling
  \item[2023 CW 49] Deploy with the notitia pilot program to collect feedback and performance metrics
  \item[2023 CW 50] Review performance analytics and feedback to improve the system
  \item[2023 CW 51] Work on minor improvements and start writing the thesis
  \item[2023 CW 52] Write thesis
  \item[2024 CW 01] Write thesis
  \item[2024 CW 02] Finalize thesis and prepare for submission
\end{description}


% \bibliographystyle{plain}
% \bibliography{literature}

\end{document}
